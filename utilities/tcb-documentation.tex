\RequirePackage{tcolorbox}
\tcbuselibrary{documentation}

\ExplSyntaxOn
\makeatletter

% #1 = tcb options
% #2 = clist of 3-tuple, "{name}{arg}{desc}, {}{}{}, ..."
\NewDocumentEnvironment{docCommands}{ O{} m }
  {
    \tcbset{#1}
    \begin{tcb@manual@entry}
    \tcb_doc_heads:n {#2}
    \nobreak\tcbset{before~ upper=}
    \kvtcb@doc@body@command@before
    \ignorespaces
  }
  {
    \ifvmode\else\unskip\fi
    \kvtcb@doc@body@command@after
    \end{tcb@manual@entry}
  }


\clist_new:N \l_tcb_doc_cmdnames_clist % list of cmd names without repetiion
\seq_new:N \l_tcb_doc_heads_seq % seq of "{{name}{para}{dect}}"

% call \tcb_doc_head:nnnn differently depending on the length of #1
% #1 = clist of 3-tuple, "{name}{par}{desc}, {}{}{}, ..."
\cs_new:Nn \tcb_doc_heads:n
  {
    % init
    \clist_clear:N \l_tcb_doc_cmdnames_clist

    \seq_set_from_clist:Nn \l_tcb_doc_heads_seq {#1}
    \seq_pop_left:NN \l_tcb_doc_heads_seq \l_tmpa_tl
    \seq_if_empty:NTF \l_tcb_doc_heads_seq
      {
        % length == 1
        \exp_after:wN \tcb_doc_head:nnnn \l_tmpa_tl {}
      }{
        % length >= 2
        \exp_after:wN \tcb_doc_head:nnnn \l_tmpa_tl
          {after~ skip=0pt, enlarge~ bottom~ by=0pt}
        \seq_pop_right:NN \l_tcb_doc_heads_seq \l_tmpa_tl
        \seq_if_empty:NF \l_tcb_doc_heads_seq
        {
          % length >= 3
          \seq_map_inline:Nn \l_tcb_doc_heads_seq
            {
              \tcb_doc_head:nnnn ##1
                {before~ skip=0pt, after~ skip=0pt, enlarge~ bottom~ by=0pt}
            }
        }
        \exp_after:wN \tcb_doc_head:nnnn \l_tmpa_tl
          {before~ skip=0pt}
      }
  }
\cs_generate_variant:Nn \tcb_doc_heads:n {V}

% execute single {tcb@doc@head} env
% #1 = command csname
% #2 = arg spec
% #3 = command description
% #4 = tcb options
\cs_new:Nn \tcb_doc_head:nnnn
  {
    \begin{tcb@doc@head}{doc@head@command, #4}
    % it is better to append \strut inside \tcb@Print@Com
    \strut
    \tcb@Print@Com{#1}
    % for the same command name, write at most one index and one label 
    \clist_if_in:NnF \l_tcb_doc_cmdnames_clist {#1}
      {
        \tcb@index@Com{#1}
        \protected@edef\@currentlabel{\noexpand\tcb@cs{#1}}
        \label{com:#1}
        \clist_gput_right:Nn \l_tcb_doc_cmdnames_clist {#1}
      }
    {\ttfamily #2}
    % more efficent than \tcbset{doc description=#3}
    \gdef\kvtcb@doc@description{#3}
    \tcb@doc@do@description
    \end{tcb@doc@head}
  }


%%
%% a key-value scheme
%%

\ExplSyntaxOff
% use consistent key & stored-in cmd names with option "doc description",
% which is defined as
%   doc description/.store in=\kvtcb@doc@description,%
\tcbset{
  doc name/.store in=\kvtcb@doc@name,
  doc parameter/.store in=\kvtcb@doc@parameter,
  % already defined
  % doc description/.store~ in=\kvtcb@doc@description,
  documentation/name/.forward to=/tcb/doc name,
  documentation/para/.forward to=/tcb/doc parameter,
  documentation/desc/.forward to=/tcb/doc description
}

% init to define stored-in commands
\tcbset{documentation/.cd, name=, para=}
\ExplSyntaxOn

% #1 = tcb options
% #2 = default cmd info, clist "name=..., para=..., desp=..."
% #3 = variant cmd info, clist info-clist "name=..., para=..., desp=..."
% see more discussion in https://github.com/T-F-S/tcolorbox/issues/89#issuecomment-570816807
\NewDocumentEnvironment{docCommands-keyvals}{ O{} m O{} }
  {
    \tcbset{#1}
    \begin{tcb@manual@entry}
    % the only difference compared to {docCommands} env
    \tcb_doc_heads_kv:nn {#2} {#3}
    \nobreak\tcbset{before~ upper=}
    \kvtcb@doc@body@command@before
    \ignorespaces
  }
  {
    \ifvmode\else\unskip\fi
    \kvtcb@doc@body@command@after
    \end{tcb@manual@entry}
  }

\clist_new:N \l_tcb_doc_heads_clist % store #1 of \tcb_doc_heads:n
\seq_new:N \l_tcb_temp_seq % store one item of \l_tcb_doc_heads_clist

\cs_new:Nn \tcb_doc_heads_kv:nn
  {
    % init
    \clist_clear:N \l_tcb_doc_heads_clist
    \clist_clear:N \l_tcb_doc_cmdnames_clist
    \tl_clear:N \kvtcb@doc@name
    \tl_clear:N \kvtcb@doc@parameter
    \tl_clear:N \kvtcb@doc@description
    % parse default cmd info, store results in \l_tcb_doc_heads_clist
    \tcb_doc_head_parse_kv:n {#1}
    % parse variant cmd info, use extra group to achive inheriting
    \clist_map_inline:nn {#2}
      { \group_begin: \tcb_doc_head_parse_kv:n {##1} \group_end: }
    % call \tcb_doc_heads:V
    \tcb_doc_heads:V { \l_tcb_doc_heads_clist }
  }

% parse kv list and append results to \l_tcb_doc_heads_clist
% #1 = kv list, "name=..., para=..., desc=..."
\cs_new:Nn \tcb_doc_head_parse_kv:n
  {
    \tcbset{documentation/.cd, #1}
    \seq_clear:N \l_tcb_temp_seq
    \clist_map_inline:nn { name, parameter, description }
      { \meta_seq_brace_put_right:Nv \l_tcb_temp_seq { kvtcb@doc@ ##1 } }
    \clist_gput_right:Nx \l_tcb_doc_heads_clist
      { \seq_use:Nn \l_tcb_temp_seq {} }
  }

% put #2 right to seq #1, with extra pair of braces
\cs_new:Nn \meta_seq_brace_put_right:Nn
  {
    \seq_put_right:Nn #1 { {#2} }
  }
\cs_generate_variant:Nn \meta_seq_brace_put_right:Nn { Nv }

\makeatother
\ExplSyntaxOff

\endinput


%% ----------------------------
%% begin: original definitions
%%   from tcbdocumentation.code.tex
%%   link https://github.com/T-F-S/tcolorbox/blob/master/tex/latex/tcolorbox/tcbdocumentation.code.tex
%% ----------------------------

\newenvironment{docCommand}[3][]{\tcbset{#1}%
  \begin{tcb@manual@entry}%
  \begin{tcb@doc@head}{doc@head@command}%
  \tcb@Print@Com{#2}\tcb@index@Com{#2}\protected@edef\@currentlabel{\noexpand\tcb@cs{#2}}\label{com:#2}{\ttfamily #3}%
  \tcb@doc@do@description%
  \end{tcb@doc@head}\nobreak\tcbset{before upper=}\kvtcb@doc@body@command@before\ignorespaces}%
  {\ifvmode\else\unskip\fi\kvtcb@doc@body@command@after\end{tcb@manual@entry}}

\newenvironment{tcb@manual@entry}{\begin{list}{}{%
  \setlength{\leftmargin}{\kvtcb@doc@left}%
  \setlength{\itemindent}{0pt}%
  \setlength{\itemsep}{0pt}%
  \setlength{\parsep}{0pt}%
  \setlength{\rightmargin}{\kvtcb@doc@right}%
  }\item}{\end{list}}

\newtcolorbox{tcb@doc@head}[1]{blank,colback=white,colframe=white,
  code={\tcbdimto\tcb@temp@grow@left{-\kvtcb@doc@indentleft}%
        \tcbdimto\tcb@temp@grow@right{-\kvtcb@doc@indentright}},
  grow to left by=\tcb@temp@grow@left,%
  grow to right by=\tcb@temp@grow@right,
  sidebyside,sidebyside align=top,
  sidebyside gap=-\tcb@w@upper@real,
  phantom=\phantomsection,%
  enlarge bottom by=-0.2\baselineskip,#1}

%% ----------------------------
%% end: original definitions
%% ----------------------------
